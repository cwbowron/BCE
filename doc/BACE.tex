\documentclass{article}
\author{Christopher W. Bowron \and Robert D. Abernethy IV}
\title{Bowron Abernethy Chess Engine}
\begin{document}
\maketitle

\begin{abstract}
In this paper we will discuss the game of chess and how the AI
community has attempted to solve the problem of creating an agent that
is capable of beating human grand masters.  We will also discuss our
chess agent, BACE, and describe the specific algorithms that we
implemented.  We will focus our attention on two main aspects of BACE
- searching and learning.  The searching algorithm consists of a
modified version of the alpha-beta search called NegaScout.
Our learning algorithm is a variation on the TD($\lambda$) technique.  A
discussion of our experiments and results will follow our description
of BACE.  Finally, we will state our conclusions and contemplate
future work. 

\end{abstract}

\section{Chess}
The game of chess is a good domain for practicing artificial
intelligence methods.  This is
mainly due to the environment.  Chess
consists of a set of states, from each state a set of legal actions
are possible.  The game can end in one of three outcomes - win, lose,
or draw.  The chess environment can be characterized as follows:

\begin{itemize}
\item Accessible - all aspects of the game are perceivable by the agent
\item Deterministic - the actions available are completely dependent on the current state
\item Non-episodic - actions taken in the game will affect later actions
\item Semi-static - the environment does not change while an agent is
thinking, however, the amount of time the agent has does.
\item Discrete - there are a finite number of possible actions at any given state
\end{itemize}

A simple strategy for building a chess agent would be to begin at the
initial state, examine all the possible moves, and search each move and
every possible counter-move until a path is found to a win state.
This is indeed a simple strategy; however, such a brute force method
is not feasible due to the large branching factor at each state.
Thus, AI strategies must be used. These strategies view the game of
chess as one large tree and they attempt to search through the tree as
efficiently as possible. 

\section{History}

The history of chess in the AI community dates back about fifty years
when Claude Shannon distinguished between two different types of chess
agents.  The first type of agent, Type-A, referred to agents that
implemented a brute force method.  Type-B, on the other hand,
described those agents that took into account some chess logic and
only examined a subset of moves at any given state.  Since the 1950's,
great strides have been occured in the AI community.  In May 1997, a
brute force agent named Deep Blue defeated the human world champion,
Gary Kasporov.  

Although there are many chess agents around, we wish to discuss one in
particular due to the influence it had in the development of BACE's
learning algorithm.  In
Baxter, Tridgell, and Weaver's paper ``Learning to Play Chess using
Temporal Differences'', they discuss a chess agent called KnightCap.
What makes KnightCap interesting and different than most chess agents
is its ability to learn its evaluation function.  It does this by
using a modified version of Sutton's TD($\lambda$) algorithm. 

The TD($\lambda$) algorithm had been used previously (and with great
success) 
by Tesauro's backgammon agent, TD-Gammon.  However, simply
implementing the algorithm in a chess agent isn't practical because
chess requires a deeper search through the tree.  In TD-Gammon the
search is very shallow - one to three ply.  To be competitive in chess
it is necessary to search as deep as at least five or six ply.
Therefore, Baxter, Tridgell, and Weaver needed to make some
modifications to the 
general algorithm.  The major modification was to have the TD
algorithm execute on the leaf states of its searches, rather than the
actual positions that occurred during the game. 


\section{BACE}

\subsection{Overview}
BACE is a fairly simple implementation of chess.  It is designed to
work in one of three modes:  independently, with the X interface XBoard, or run
remotely on a server using RoboFics.  The core chess engine
was written by Christopher Bowron.  The learning algorithms were
implemented by Christopher Bowron and Robert Abernethy.  Table
\ref{files} in the appendix contains locations of the files.

While being a relatively simple implementation, it has a number of
powerful features, including transposition tables, NegaScout search,
move ordering, and a temporal difference learning algorithm.

\subsubsection{Board Representation}

The basic board representation in BACE is an array of 64 integers.
Each array item is a mapping between a chess piece and an integer
between 0 and 13.  The high order bits of the integer represent the
piece.  0 represents an open square, 1 represents pawn, and so on,
with 6 representing a king.  The lowest order bit represents the piece
colors.  0 represents white, 1 represents black.  
This representation was chosen to allow for quick testing of a squares
occupancy as well as determine if the piece are friend or foe.  

Along with the basic square information, some additional information
is updated dynamically.  One such feature is the flags integer which
consists of 12 bits 
enumerating castling rights, en passantes, double pawn pushes,
castling information, and promotional information.

An array representing the number of pawns on each file is also kept
dynamically, as well as the material each side has, a positional
evaluation of each sides pieces, an integer representing the current
board hash value, the number of pieces on the board, an array
containing the number of each individual piece on the board, and also
the squares on which the kings reside.  This information is kept to
facilitate evaluations, move generations and move legality checking.  

\subsection{Search Algorithms}

Implemented in BACE are many different variants of the negamax search
algorithm.  By default, BACE uses the NegaScout algorithm invented by
Prof. Dr. Alexander Reinefeld which is a
modified alpha-beta pruning technique which greatly reduces the number
of nodes searched.  The main idea is to search with a null window in
most of the nodes and only with a wider window where it is
necessary. If the search requires the full window the position must be
re-searched. Thus, move ordering and storing encountered positions are
very important when using NegaScout.

The basic NegaScout algorithm is:

\begin{verbatim}
int NegaScout ( position p; int alpha, beta );   
{             
   int a, b, t, i;

   determine successors p_1,...,p_w of p;

   if ( w == 0 )
      return ( Evaluate(p) );                   /* leaf node */

   a = alpha;
   b = beta;
   for ( i = 1; i <= w; i++ )
   {
      t = -NegaScout ( p_i, -b, -a );
      if (t > a) && (t < beta) && (i > 1) && (d < maxdepth-1)
         a = -NegaScout ( p_i, -beta, -t );     /* re-search */

      a = max( a, t );
      if ( a == beta ) 
         return ( a );                            /* cut-off */
      b = a + 1;                      /* set new null window */
   }
   return ( a );
}
\end{verbatim}

Also implemented is a generic alpha-beta pruning search, as well as a
full negamax search. 

After doing a full search to the specified depth, BACE uses a quiescent
search to find positions that are relatively stable to facilitate
better evaluations.

BACE uses an iterative deepening search algorithm on top of the other
search.  The search depth begins at the previously stored depth based
on the transposition table entries.  The search is made at the current
depth, then increased and searched again, and so on, until the search
limit is reached.  The search can be limited by time, depth, or both.   

An aspiration window can also be set up but was not used in the course
of these experiments.  Using the aspiration search, The alpha and beta
values are initialized to the evaluation of the current position minus
the value of two pawns for alpha, and plus two pawns for beta.  If our
search results in a number between those bounds, it is valid.  If it
lies outside that range, we must search again with the full window.
After each iteration, the alpha and beta values are updated based on
the evaluation at the previous depth. 

\subsection{Move Ordering}

Move Ordering is very important when using alpha-beta pruning
techniques.  Therefore, BACE uses a number of different criteria for
sorting moves.

(To scale numeric values, assume that a pawn has a value of 1,000.)

The first criteria used in ordering moves is the difference in
material value.  A bonus is given for captures of 1,000 plus 300 times
the ratio of capturer to capturee.  Thus, the bonus for a pawn
capturing a queen is very high, and a queen capturing a pawn is much
lower.  Non-capture moves are initialized to zero.

Bonuses are awarded to the base result in the following way:

\begin{itemize}

\item If the move has been stored as a refutation of the last move,
a 2,000 point bonus is awarded.   

\item If the position is stored in the transposition table, a 1,500
point bonus is awarded to the stored move for that position. 
	
\item If the move has been stored as the primary 'killer move' for
the current ply, a 1,000 point bonus is awarded.

\item If the move has been stored as the secondary 'killer move' for
the current ply, a 500 point bonus is awarded.

\item Finally, an array called history is kept that is maintained as
a count of the number of times a move resulted in the changing of the
alpha-beta bounds.  The value of that array for this move is added to
the value of this move. 

\end{itemize}

Once each move has been evaluated according to these criteria, it is
sorted using a QuickSort algorithm.

\subsection{Book}

BACE uses the book from Marcel's Simple Chess Program, written by
Marcel van Kervinck.  The book consists of around 800 openings, of
which BACE will select the lines consistant with the current game and
randomly select  
one line for its next move.  Once there are no longer available
openings, BACE will begin using its searching algorithms for move
selection.  

\subsection{Transposition Tables}

By default, BACE uses a transposition table of $2^(20)$ entries.  Each
entry consists of an integer hash value, an integer depth, a flag, a
score and move.  The flag is used to signify whether the value stored
is an upper bound, a lower bound, or the exact result of evaluation
this position searching to the specified depth.  The transposition
table greatly reduces the number of positions that need to be
searched, because once an encountered position is reached, the score
from the table can be used rather than continuing the search.
Positions are located in the table by indexing into the transposition
table array based on the lower order bits of the current board hash
value.  If the value stored in the hash value in the table is equal to
the current board hash value, it is assumed that the stored position
is the same as the current position and the table can be used.

\subsection{Evaluation Features}

The evaluation function uses a number of different criteria.  The
score is first computed for white.  Points are added for each white
feature on the board, and subtracted for each black feature.  If the
evaluation is used to determine how well black is doing, the result is
negated.

Each piece has a weight associated with it.  Originally, a pawn was
1000, a knight 3250, a bishop 3500, a rook 5000, and queen 9000.  The
king was not given a value because the loss of a king is the loss of a
game.

Each piece has an associated array of 64 integers.  Each entry in the
array is a bonus for that piece to be on that square.  This array
promotes the advancement of pawns, and central control for knights,
bishops, rooks, and queens.

Checkmate and stalemate are not recognized by the evaluation function,
rather they are evaluated inside the search function when a side to
move is found to have no valid course of action.  However, if there is
not enough material for either side to mate, the draw is recognized by
the evaluator.  If one side does not have potential mating material
and the other does, the losing side is penalized by a weight called
"NOMATERIAL".

Table \ref{eval} contains the a summary of the additional evaluation
features. 

\def\feature#1#2{{\em #1} & #2 \\ \hline}

\begin{table}[hpbf]
\caption{Evaluation Features}
\label{eval}
\begin{center}
\begin{tabular}{|lp{8cm}|}

\hline

\feature{KCASTLEBONUS}{Bonus for having castled on the king's side.}
        
\feature{QCASTLEBONUS}{Bonus for having castled on the queen's side.}
        
\feature{NOCASTLEQUEEN}{Penalty for not having castled, and losing the
right to castle queen side.}
        
\feature{NOCASTLEKING}{Penalty for not having castled, and losing the
        right to castle king side.}

\feature{QUEEN\_TROPISM}{
        Bonus for having queen close to opponent's king.  This bonus
        is multiplied by the minimum of the difference of the queen's
        rank and the opponents king's rank and the difference of their
        files.}
        
\feature{ROOK\_TROPISM}{
        Bonus for having rook close to opponent's king.  This bonus
        is multiplied by the minimum of the difference of the queen's
        rank and the opponents king's rank and the difference of their
        files.}
    
\feature{DOUBLEDROOKS}{
        Bonus for having two rooks on the same rank or file.}
        
\feature{OPENFILE}{
        Bonus for having a rook on a file that contains no friendly
        pawns or opponent's pawns.}
        
\feature{SEMIOPEN}{
        Bonus for having a rook on a file that contains no friendly
        pawns.}
        
\feature{TWOBISHOPS}{
        Bonus for having two or more bishops.}
        
\feature{KNIGHT\_TROPISM}{
        Bonus for having knight close to opponent's king.  The bonus 
        is calculated as this number multiplied by the addition of the 
        rank difference and the file difference.}
        
\feature{ISOLATED}{
        Penalty for pawns that have no friendly pawns on the adjacent
        squares.}
        
\feature{DOUBLED}{
        Penalty for having multiple pawns on a file.  Scored once for
        each pawn on the file.}
        
\feature{BACKEDUP}{Penalty for a pawn that is behind adjacent friendly
pawns.} 

\feature{NOMATERIAL}{Penalty for not having enough material to mate.
        This is defined as having no more than one knight or one
        bishop.}
        
\feature{SEVENTH\_RANK\_ROOK}{Bonus for having rook on its seventh
        rank.  That is white rooks on the seventh rank and black rooks
        on the second rank.} 

\hline

\end{tabular}
\end{center}
\end{table}

\subsection{Learning Algorithm}

The learning algorithm is a modification of TD($\lambda$) known as
TDLeaf.  It was introduced by Baxter, et. al. The main
difference between TD($\lambda$) and TDLeaf is that the leaf nodes that
were actually evaluated are those on which the temporal differences
algorithm is used. 

To implement TDLeaf, we modified our quiescent search to save the
board of positions that evaluate between the alpha and beta bounds.
After the search is completed BACE checks to see if the result of the
search is the same as that stored.  If not it checks to see if the
value stored is the same as the previous value stored.  If it is, BACE
will use the stored board from the previous evaluation.  If neither of
these boards are correct, BACE checks the principle variation returned
to see if it leads the the correct board evaluation.  If so, it is
stored, if not the position is ignored when calculating the temporal
differences.  These steps were necessary because the use of the
transposition tables means that sometimes searches are not actually
carried out fully to the leaf nodes.  We used this approach to
decrease the amount of necessary overhead for board storage and
copying.

At the end of the game, each weight was updated according to the
TD($\lambda$) algorithm (see \ref{tdlambda}).  For our experiments we
used a $\lambda$ value 
of 0.70 and a learning rate, $\alpha$ of 0.70.

\begin{equation}
w := w + \alpha \sum_{t=0}^n (E_{t+1} - E_t) \sum_{k=0}^t
\lambda^{t-k} \nabla_w E_t
\label{tdlambda}
\end{equation}


To decrease the fluctuations encountered in evaluations, we used the
$tanh$ function to smooth our values.  The temporal differences used
where based on the values of $tanh(\beta E_t)$ where $\beta$ was a value
chosen so that an evaluation of up one pawn was equal to .25.

The actual code used to implement the TD learning was as follows:
\begin{verbatim}
for (w = PAWN_VALUE + 1; w < LAST_WEIGHT; w ++)
{
    for (p = 1; p <= (n-1);p++)
    {
        int j;
        double S2 = 0.0;
        
        for (j = 1; j<= p; j++)
        {
            double grad =
                    (1.0 - tanhvector[j] * tanhvector[j]) *
                    EVAL_SCALE * gradients[w][j];
                    
            S2 += pow(TD_LAMBDA, p-j) * grad;
        }
        delta[w] += TD_ALPHA * d[p] * (S2 / EVAL_SCALE);
    }
        
    weights[w] += delta[w];
}   
\end{verbatim}
The $1 - tanh^2 (\beta E_t)$ was used instead of directly calculating
$sech^2 (\beta E_t)$
because it was faster to use the values we had already computed.

\section{Experiments}

During the development and debugging stage, BACE was tested using
self-play to ensure that the values arrived at through the temporal
differences was reasonable.  Once we were fairly certain things were
running smoothly, we reset the weights to their original states.  No
experiments were done to compare how a self-taught BACE compared with
with an online taught BACE, although this may be an interesting
experiment to run.  

We placed BACE on the Free Internet Chess Server (freechess.org) with
its initial weights.  BACE was set to accept only rated 5 minute
games.  Initially, its rating was around 1300.  Before implementing
the learning features its rating was around 1400-1500.

\section{Results}

With our learning algorithm, BACE's rating rose from around 1300 to a
maximum of 1660, over a period of 24 hours, playing approximately 200
games.  For comparison, Master level rankings begin at 2,000.  

% % insert graph
% \begin{figure}[hbf]
% \centering
% \input{rank}
% \caption{graph depicting rating fluctuation}
% \label{rankgraph}
% \end{figure}

\section{Conclusions}

We feel that BACE performed reasonably well, despite the fact that
it's rating dropped after implementing our learning algorithm.  Having
a static evaluation function with predetermined weights, BACE was
able to achieve a rating of approximately 1500.  After implementing
our learning algorithm, its rating dropped down to about 1300.  We
expected this initial drop due to the fact that BACE would not be able
to search as far as previously because of the  increased overhead that
was necessary to store the extra board 
positions needed by our learning algorithm.  This overhead took up
valuable time and limited the depth of our searching algorithm.  We
predicted an increase in our 
rating as the number of games played increased and our agent was able
to learn a more accurate evaluation function.  As expected BACE's
rating increased but was limited by the small set of evaluation
features.

\section{Future Work}

Although BACE enjoyed modest success, we believe that some minor
changes would be beneficial.  One  
possible adjustment would be to add more evaluation features.  BACE
uses very few features in its evaluation compared to many other
engines.  Another possible improvement would be to move the currently
hard-coded positional arrays into the weights that are learned based
on temporal differences.  This could have a significant effect on
BACE's positional play.  The way in which features are updated by
temporal difference make it easy to add more features.

Another possible approach likely to increase level of play would be to
split the evaluation weights into more than one set of values.  We
believe that separating 
weights into two, or possibly three stages, beginning, middle and
endgame may be useful.

Once the weights begin to converge, it may be a useful feature to
learn which openings BACE is successful with and which it is not.
This could be done by recording information on which openings have
led to wins, and selecting those with a higher probability in the
opening moves.

\appendix

\section{Related Work}
\begin{enumerate}

\item
Tridgell, A., Baxter J., \& Weaver L. Learning To Play Chess Using
Temporal Differences,{\em Machine Learning, 40},243-263.
\item
Sutton, Richard. Learning to Predict by the Methods of Temporal
Differences. {\em Machine Learning, 3},39-44.  Richard Sutton. 
\item
Samuel, A. L. Some studies in machine learning using the game of
checkers.  {\em IBM Journal of Research and Development, 3}, 210-229. 
\end{enumerate}


\section{File Locations}

\begin{table}[hpbf]
\begin{center}
\begin{tabular}{|ll|}
\hline
BACE &
{\em http://www.cse.msu.edu/~bowronch/FILES/bce.tar.gz} \\
XBoard &
{\em http://research.compaq.com/SRC/personal/mann/xboard.html} \\
RoboFics &
{\em http://www.freechess.org/~hawk/robofics.html} \\
mscp &
ftp://ftp.freechess.org/pub/chess/Unix/mscp-1.0.tar.gz \\
\hline

\end{tabular}
\end{center}
\caption{file locations}
\label{files}
\end{table}

\end{document}
